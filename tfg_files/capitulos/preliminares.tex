\chapter*{Resumen}
\label{Resumen}

El proyecto consiste en una herramienta para simular escenarios 
bélicos al estilo de los que podemos encontrar en juegos del género \textit{\ac{RTS}} como son
\textit{`\ac{TaB}'} o la saga \textit{`\ac{AoE}'}, desarrollado
completamente en C++ para \ac{PC}.

La ``demo'' se compone de un escenario, una serie de unidades controladas por el jugador
que conforman el ejército bajo sus órdenes y por último, una serie de objetivos enemigos
que deberemos abatir. Una vez alcanzado el reto propuesto, el juego termina dando la opción 
al jugador entre repetir el nivel o volver al menú principal.

\chapter*{Justificación y objetivos}
A lo largo de la carrera son muchas las asignaturas que requieren y en las que se desarrollan
habilidades relacionadas con la programación en diversos lenguajes y usos, 
pero no es hasta el tercer año que se me presenta la oportunidad de desarrollar
un videojuego completo. 

Durante la asignatura de `Fundamentos de los Videojuegos' tuve por primera vez
la experiencia de enfrentarme al desafio que supone crear un videojuego desde cero.
Fue en ese momento cuando me di cuenta de que a lo que me quería dedicar es a
desarrollar juegos de forma profesional. A lo largo del cuarto curso y junto a los demás
integrantes de \textit{'Sunlight Studio'} desarrollamos \textit{'Cyborgeddon'}. Este proyecto
también contribuyó a que decidiera que es a los videojuegos a lo que quiero dedicar mi futuro.

A lo largo de mi vida he jugado a una considerable cantidad de juegos entre los cuales
se puede apreciar una inclinación por los juegos de aventura y exploración, los 
\ac{RPG} y los \ac{RTS} o de estrategia en general. Precisamente por esto que me hace especial
ilusión desarrollar un juego de uno de estos géneros.\\
Teniendo en cuenta esto la elección de desarrollar un juego del tipo \ac{RTS} para la
realización del \ac{TFG} se debe a que, bajo mi criterio, es el género con mayor
potencial para ayudarme a mejorar mis habilidades como programador, ya que se basa en
unas mecánicas con las que no he trabajado con anterioridad.

Una vez dicho esto, los objetivos planteados para el proyecto son los siguientes:
\begin{itemize}
	\item Aumentar mis conocimientos de C++
	\item Comprender mejor el funcionamiento del \ac{PC}
	\item Aprender nuevas técnicas de \ac{IA}
	\item Profundizar en la arquitectura \ac{ECS}
	\item Desarrollar un producto mediante el cual mostrar mis habilidades
\end{itemize}

\cleardoublepage %salta a nueva página impar
% Aquí va la dedicatoria si la hubiese. Si no, comentar la(s) linea(s) siguientes
\chapter*{Agradecimientos}
\setlength{\leftmargin}{0.5\textwidth}
\setlength{\parsep}{0cm}
\addtolength{\topsep}{0.5cm}
\begin{flushright}
\small\em{
A mi madre Gabriela,\\
por el increíble esfuerzo que realiza cada día por mi. 
}
\end{flushright}
\begin{flushright}
\small\em{
A mi hermana Raquel,\\
por haber sido y ser un faro para mi. 
}
\end{flushright}
\begin{flushright}
\small\em{
A mi buen compañero Jorge Espinosa,\\
por nuestra convivencia estos años. 
}
\end{flushright}

\cleardoublepage %salta a nueva página impar
% Aquí va la cita célebre si la hubiese. Si no, comentar la(s) linea(s) siguientes
\chapter*{}
\setlength{\leftmargin}{0.5\textwidth}
\setlength{\parsep}{0cm}
\addtolength{\topsep}{0.5cm}
\begin{flushright}
\small\em{
Para saber hablar es preciso saber escuchar\\
}
\end{flushright}
\begin{flushright}
\small{
Plutarco.
}
\end{flushright}
\cleardoublepage %salta a nueva página impar
