\chapter*{Resumen}
\label{Resumen}
%\thispagestyle{empty}
\textbf{titulo por definir} es una herramienta para simular escenarios 
bélicos como los que podemos encontrar en juegos del género \ac{RTS} como son
\textit{`They are Billions'} o la saga \textit{`Age of Empires'}, desarrollado
completamente en C++ para \ac{PC}.

La simulación se compone de un escenario, una unidad controlada por el jugador,
una serie de unidades que conforman el ejercito bajo sus ordenes y por último, 
tenemos una serie de objetivos que abatir. Una vez conseguido el objetivo terminará
el juego pudiendo el jugador elegir entre repetir el nivel o volver al menú principal.

\chapter*{Justificación y objetivos}
A lo largo de la carrera son muchas las asignaturas que requieren y desarrollan
habilidades relacionadas con la programación en diversos lenguajes y usos, 
pero no es hasta el tercer año que se me presenta la oportunidad de desarrollar
un videojuego completo. 

Durante la asignatura de `Fundamentos de los Videojuegos' tuve por primera vez
la experiencia de enfrentarme al desafio que es crear un videojuego, y fue en ese
momento cuando me di cuenta de que a lo que me quería dedicar es a
desarrollar juegos de forma profesional. A lo largo del cuarto curso junto a los demás
integrantes de \textit{'Sunlight Studio'} desarrollamos \textit{'Cyborgeddon'}, y fue
en este momento que terminé de decidir que es a lo quiero dedicarme en el futuro.

A lo largo de mi vida he jugando una considerable cantidad de juegos entre los cuales
se puede apreciar una inclinación por los juegos de aventura y exploración, los 
\ac{RPG} y los \ac{RTS} o de estrategia en general, es por esto que me hace especial
ilusión desarrollar un juego de uno de estos géneros.\\
Teniendo en cuenta esto la elección de desarrollar un juego del tipo \ac{RTS} para la
realización del \ac{TFG} se debe a que según mi criterio es el que tiene mayor
potencial para ayudarme a mejorar mis habilidades como programador, ya que se basa en
unas mecánicas con las que no he trabajado con anterioridad.

Una vez dicho esto, los objetivos planteados para el proyecto son los siguientes:
\begin{itemize}
	\item Aumentar mis conocimientos de C++
	\item Comprender mejor el funcionamiento del \ac{PC}
	\item Aprender nuevas técnicas de \ac{IA}
	\item Desarrollar un producto mediante el cual mostrar mis habilidades
\end{itemize}

\cleardoublepage %salta a nueva página impar
% Aquí va la dedicatoria si la hubiese. Si no, comentar la(s) linea(s) siguientes
\chapter*{Agradecimientos}
\setlength{\leftmargin}{0.5\textwidth}
\setlength{\parsep}{0cm}
\addtolength{\topsep}{0.5cm}
\begin{flushright}
\small\em{
A mi madre Gabriela,\\
por el increíble esfuerzo que realiza cada día por mi. 
}
\end{flushright}
\begin{flushright}
\small\em{
A mi hermana Raquel,\\
por haber sido y ser un faro para mi. 
}
\end{flushright}
\begin{flushright}
\small\em{
A mi buen compañero Jorge Espinosa,\\
por nuestra convivencia estos años. 
}
\end{flushright}
\begin{flushright}
\small\em{
A mi querido amigo Guillermo y su familia,\\
por el cariño recibido incluso a pesar de la distancia. 
}
\end{flushright}

\cleardoublepage %salta a nueva página impar
% Aquí va la cita célebre si la hubiese. Si no, comentar la(s) linea(s) siguientes
\chapter*{}
\setlength{\leftmargin}{0.5\textwidth}
\setlength{\parsep}{0cm}
\addtolength{\topsep}{0.5cm}
\begin{flushright}
\small\em{
Ni los reyes ni los gobernantes llevan el cetro,\\
sino los que saben mandar\\ 
}
\end{flushright}
\begin{flushright}
\small{
Sócrates.
}
\end{flushright}

\begin{flushright}
\small\em{
Para saber hablar es preciso saber escuchar\\
}
\end{flushright}
\begin{flushright}
\small{
Plutarco.
}
\end{flushright}
\cleardoublepage %salta a nueva página impar
