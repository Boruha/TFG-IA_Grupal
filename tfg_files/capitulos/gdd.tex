\chapter{Documento de Diseño del Juego (GDD)}
\section{Características}
\begin{itemize}
	\item \textbf{Título:} N/A. \LaTeX \LaTeX \LaTeX \LaTeX
	\item \textbf{Plataforma:} \ac{PC} (Windows y Linux).
	\item \textbf{Género:} \acf{RTS}.
	\item \textbf{Idioma:} Inglés.
	\item \textbf{Clasificación:} PEGI 7\footnote{Web de la asociación https://pegi.info}
\end{itemize}

\section{Mecánicas}
La finalidad del juego es la de completar los distintos niveles de forma satisfactoria,
esto sucederá cuando eliminemos a todas las unidades enemigas desplegadas a lo largo del
nivel. Como herramienta para alcanzar este objetivo dispondremos de un ejercito a nuestro
mando el cual deberemos gestionar de forma efectiva para sortear los obstáculos y
desafios propuestos.

El jugador perderá cuando todas sus unidades mueran.

\subsection{Unidades}
Entre las unidades podemos encontrar diferentes arquetipos con carácteristicas propias
que nos permitiran crear variedad en las posibles soluciones a la hora de superar el
nivel.

Los distintos tipos son los siguientes:
\begin{itemize}
	\item \textbf{Soldado:} es la unidad más básica que podemos encontrar en el campo de
							guerra, esta armado con una espada y posee estadisticas
							bajas.
	\item \textbf{Arquero:} van equipado con arco y flechas para atacar a distancia a
							sus rivales, tiene menos resistencia que los soldados por
							lo que tendremos que protegerlos para asegurar su
							supervivencia.  
\end{itemize}
