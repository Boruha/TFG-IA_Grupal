\chapter{Conclusiones}
Para hablar de las conclusiones obtenidas del desarrollo del proyecto, quiero traer de vuelta los
objetivos que se propusieron en su momento. 

Uno de los objetivos más generales era ``Comprender mejor el funcionamiento del \ac{PC}'', gracias a
tareas como el desarollo del tipo numérico en coma fija hemos ampliado nuestros conocimientos sobre como
traduce y codifíca el código el compilador. \\ Además, tras realizar de nuevo el proceso de preparar el
proyecto para no depender del sistema para su ejecución, hemos afianzado nuestros conocimientos sobre las
opciones de compilación y enlace de librerías.

A lo largo de estos meses la gestión de la memoria usada durante la ejecución del programa ha jugado una
carta importante. Las operaciones de reserva, liberación y acceso de memoria suponen recursos y tiempo
para sistema, hacer un uso eficiente y escoger el momento adecuado para ellas es altamente importante.
Para lograr hacer un uso más efectivo, hemos tenido que trabajar otro de los objetivos del proyecto,
``Profundizar en la arquitectura \ac{ECS}'' el cuál nos ha permitido desarrollar un pequeño motor que
gestione de forma correcta la información de nuestras entidades.

Uno de los puntos claves era estudiar y aprender nuevas técnicas de \ac{IA}, en concreto el
\textit{Flocking} y el uso de los \textit{Steering Behaviors}, además de poder crear una demo que nos
permite mostrar el conocimiento adquirido. \\
Gracias a la lectura a conciencia sobre el tema y la dedicación de un largo periodo de tiempo para
desarollar algoritmos y funciones que trabaja estas técnicas, hemos conseguido entender su
funcionamiento a la perfección y obtener resultados que satisfacen este altamente objetivo.

Al desarrollar todo únicamente en \textit{C++17}, hemos conseguido apliar el conocimiento sobre el lenguaje
y entender mejor el funcionamiento de algunas funcionalidades y herramientas incluidas en la \textit{STL} y
a desarrollar las nuestras propias, como puede ser el ya mencionado tipo de dato en coma fija, siguiendo una
serie de normas y reglas estandarizadas.


\section{Lecciones aprendidas}
Después de estos meses trabajando en el \ac{TFG}, hay una serie de mensajes y/o ideas que han ganado peso y
que es puede ser interesante tener en cuenta.

A la hora de desarrollar un proyecto, herramientas y códigos hay miles, pero lo que siempre vas a tener es
una máquina que lo interprete y ejecute. Conocer la máquina y como gestiona tu trabajo te permitirá ser más óptimo y encontrar fallos más rápido de haberlos, por ello es importante invertir tiempo en entender como
funciona a bajo nivel. 

Un pilar fundamental de la ingeniería son las matemáticas, es importante no descuidarlas y entender como
hacer un uso correcto de ellas aplicado a la informática. Una vez más, entendiendo como las gestiona y
representa internamente.

Es importante planificar el desarrollo con periódos de tiempo realistas, las prisas y la mala planificación
llevan a soluciones temporales y malos funcionamientos.

Cuanto más general y flexible sea el código, mejor. Diseñar las herramientas de forma que sean reutilizable
te facilitará el trabajo futuro y te hará plantear de mejor forma su funcionamiento.

Internet cuenta con mucha información y gente dispuesta a ayudar, pero no hay que olvidarse de los libros
como fuente de un conocimiento más completo y profundo.


\section{Trabajos futuros}
Con el fin de evaluar que aspectos del proyecto quedan por finalizar, vamos a elaborar dos
listas con elementos técnicos y de diseño que serían deseables para en el proyecto.
La primera serán elementos que estimamos de alta prioridad y necesarios, que por motivos de tiempo
y desarrollo no se han posido integrar a tiempo, la segunda serań cuestiones que es altamente probable 
que no se puedan realizar, que tienen una baja prioridad por el momento o que estan fuera de los
objetivos generales del trabajo y por eso se han quedado fuera del desarrollo realizado hasta la fecha.

\textbf{Objetivos principales:}
\begin{itemize}
	\item Dar \textit{feedback} al jugador del estado de las unidades en el nivel.

	\item Crear un sistema de menús para la ejecución del juego.
\end{itemize}

\textbf{Objetivos deseables:}
\begin{itemize}
	\item Ampliar el uso de OpenGL e intergrar sistema de carga de imagenes para poder añadir
	elementos visuales a las entidades, escenario y menús.

	\item Obtener/diseñar recursos gráficos apropiados para el proyecto.

	\item Integrar un sistema de sonido para poder reproducir al menos una melodia.
\end{itemize}