\chapter{Conclusiones}
Er final primo.

\section{Lecciones aprendidas}
hablar de los hitos/cosas importantes en general

\section{Trabajos futuros}
Con el fin de evaluar que aspectos del proyecto quedan por finalizar, vamos a elaborar dos
listas con elementos técnicos y de diseño que sería deseable poder introducir en el proyecto.
La primera serán elementos que estimamos posibles de alcanzar en el tiempo restante de
desarrollo, y la segunda cuestiones que es altamente probable que no se puedan acometer.

\textbf{Objetivos principales:}
\begin{itemize}
	\item Plantear como dar \textit{feedback} al jugador del estado de las unidades en el nivel.

	\item Diseñar sistema de triggers para poder lanzar eventos para el jugador.

	\item Diseñar nivel tutorial donde se explica como jugar e interactuar con las unidades
	tanto aliadas como enemigas y las reglas.

	\item Crear un menú de inicio previo a la ejecución del juego.

	\item Redacción completa de la memoria.

	\item Montar el proyecto para compilar tanto en Windows como en Linux.
\end{itemize}

\textbf{Objetivos deseables:}
\begin{itemize}
	\item Ampliar el uso de OpenGL e intergrar sistema de carga de imagenes para poder añadir
	elementos visuales a las entidades, escenario y menús.

	\item Obtener/diseñar recursos gráficos apropiados para el proyecto.

	\item Integrar un sistema de sonido para poder reproducir al menos una melodia.
\end{itemize}