\chapter{Desarrollo y fases del producto}
\section{Iteraciones}
En esta sección se comentarán brevemente los objetivos planteados y las tareas que
se han llevado a cabo en cada iteración adjuntando el tablero de `Trello' asociado.

\subsection{Iteración 0}
Esta iteración tenía como proposito terminar de definir la idea para el
proyecto, ya que, todavía era un poco difusa la imagen del producto final que se quería desarrollar.
Para ir entrando en una dinámica productiva e ir definiendo lo que se quería hacer,
comenzamos por iniciar el desarrollo de un propotipo a la vez que preparabamos un poco
los materiales relacionados con la memoria, como puede ser la lectura de las directrices
y/o la revisión de la plantilla de \LaTeX~\ref{img:it_0}.

Al final de esta iteración el prototipo contaba con una versión básica del bucle principal
del juego donde creamos una serie de sistemas, \textit{managers} y entidades además del dibujado
y movimiento\footnote{En esta versión nos limitamos a un ``Goto'' a través de 4 puntos.} de 
estas.

\begin{figure}[ht]
\centering
\includegraphics[width=0.45\textwidth]{imagenes/metodologia/tareas_it0.png}
\caption{Lista de tareas realizadas en la iteración 0}
\label{img:it_0}
\end{figure}

\subsection{Iteración 1}
A lo largo de esta iteración seguimos añadiendo sistemas al prototipo como puede ser el
de \textit{Input} el cual nos permitirá interactuar con el producto, además de herramietas
como el mapeado del teclado o el tipo de dato en coma fija en el cual profundizaremos más tarde.

Por otro lado, se comenzó a trabajar en la \ac{IA} del juego creando los primeros comportamientos
y herramientas para cambiar entre ello. Como veremos más adelante también, aquí caeremos en los primeros
errores de concepto a la hora de trabajar con los \textit{`Steering behaviors'}.

Por último, finalizamos la introducción en el uso de \LaTeX, \textit{`JabRef'} y la plantilla comenzando 
así con la redacción de esta memoria~\ref{img:it_1}.

\begin{figure}[ht]
\centering
\includegraphics[width=0.45\textwidth]{imagenes/metodologia/tareas_it1.png}
\caption{Lista de tareas realizadas en la iteración 1}
\label{img:it_1}
\end{figure}

\subsection{Iteración 2}
Durante esta tercera iteración los objetivos eran los de corregir el funcionamiento de los
\textit{`Steering behaviors'}, añadir elementos visuales que ayuden al debug de la \ac{IA}
mostrando las componentes del movimiento de las entidades y herramientas para el control del
tiempo de ejecución y el tamaño del \textit{`DeltaTime'}.

Como podemos apreciar en la lista~\ref{img:it_2}, las tareas de control de la ejecución del
programa no fueron realizadas a tiempo, en su lugar se le dió prioridad a mejorar el uso
de la coma fija para un código más limpio y crear metodos auxiliares como puede ser el paso de
coordenadas continuas a pixeles en pantalla, permitiendo así trabajar a lo largo del programa 
usando enteros con signo y procesarlos en el sistema de \textit{`Render'}.

En lo referente a la memoria comenzamos la redacción de una versión preliminar del \ac{GDD},
Estado del Arte y metodología. 

\begin{figure}[ht]
\centering
\includegraphics[width=0.65\textwidth]{imagenes/metodologia/tareas_it2.png}
\caption{Lista de tareas realizadas en la iteración 2}
\label{img:it_2}
\end{figure}