\chapter{Desarrollo y fases del producto}
A lo largo de esta sección comentaremos todo el proceso de desarrollo del prototipo y de la 
memoria, las diferentes iteraciones del producto serán agrupadas en diversas fases con el fin
de facilitar la lectura y el entendimiento de los objetivos planteados en cada momento.   

\section{Fase 0: Definición del proyecto y primeros pasos}
Esta primera fase engloba las tres primeras iteraciones donde comenzamos a terminar de definir 
la idea para el proyecto, ya que, todavía era un poco difusa la imagen del producto final que se
quería desarrollar. \\
Para ir entrando en una dinámica productiva iniciamos el desarrollo de 
funcionalidades básicas como pueden ser el bucle principal del juego y la intergración de `TinyPTC'
en el sistema de dibujado para poder abrir una ventana y pintar \textit{sprites} básicos. Además,
la inclusión de una serie de componentes para poder hacer que las entidades siguieran una patrulla 
por el mapa.\\
Por último la preparación de las herramientas de \LaTeX y `Jabref' para redactar la memoria y
la lectura de las directrices, normas y memorias de compañeros del año pasado.

A lo largo de la iteración siguiente, seguimos añadiendo sistemas al prototipo como puede ser el
de \textit{Input} el cual nos permitirá interactuar con el producto, además de herramietas
como el mapeado del teclado o el tipo de dato en coma fija en el cual profundizaremos más tarde.\\
Por otro lado, se comenzó a trabajar en la \ac{IA} del juego creando los primeros comportamientos
y herramientas para cambiar entre ello. Como veremos más adelante también, aquí caeremos en los primeros
errores de concepto a la hora de trabajar con los \textit{`Steering behaviors'}.\\
Por último, finalizamos la introducción en el uso de \LaTeX, `JabRef' y la plantilla comenzando 
así con la redacción de esta memoria.

Durante esta tercera iteración los objetivos eran los de corregir el funcionamiento de los
\textit{`Steering behaviors'}, añadir elementos visuales que ayuden al debug de la \ac{IA}
mostrando las componentes del movimiento de las entidades, y herramientas para el control del
tiempo de ejecución y el tamaño del \textit{`DeltaTime'}.\\
Con el fin de tener un código más limpio y legible, los últimos esfuerzos de la iteración se
destinaron a mejorar el uso de la coma fija y crear metodos auxiliares como puede ser el paso de
coordenadas continuas a pixeles en pantalla, permitiendo así trabajar a lo largo del programa 
usando enteros con signo y procesarlos en el sistema de \textit{`Render'} para adaptar los datos
al dibujado.\\
En lo referente a la memoria comenzamos la redacción de una versión preliminar del \ac{GDD}
donde comenzamos a describir las características del producto, del Estado del Arte donde se hace
mención a juegos que nos han servido de inspiración o modelo para imitar y se incluye una explicación
sobre las ténicas y algotimos se van a usar, y la explicación sobre la metodología seguida durante
este desarollo. 

\section{Fase 1: Producto mínimo viable}
En esta segunda fase el objetivo era el de alcanzar los mínimos aceptables tanto para el prototipo
como para la memoria, en el caso del prototipo esto significa implementar todas las funcionalidades
recogidas en el `Producto mínimo viable', y en el caso de la memoria significa tener una versión
que contenga toda la estructura y contenido deseado (a falta de realizar las correcciones
pertinentes).

La fase comenzó con una iteración corta debido a las vacaciones de Navidad y final de año, en la
que el foco estuvo despositado en refactorizar y dejar en mejor estado el código desarrollado
hasta el momento, donde se separó el nucleo principal del motor \ac{ECS} del conjunto de sistemas y
herramientas en dos \textit{namespaces} diferentes. Además, se realizaron cambios en el diseño
de las entitades y la fachada usadas.

