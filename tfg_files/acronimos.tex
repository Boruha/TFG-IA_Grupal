\chapter*{Índice de Acrónimos}
\label{acronimos}
% fichero para poner los acrónimos
%para escribir un acrónimo nuevo, seguir el siguiente protocolo: (POR EJEMPLO PARA PONER EL ACRÓNIMO IEEE)
%- En el texto, poned \ac{IEEE}
%- En el fichero acrónimos.tex, poner la siguiente enTrada:
% \acrodef{IEEE}{Institute of Electrical and Electronics Engineers}
% IEEE: Institute of Electrical and Electronics Engineers

%Además de \ac{IEEE} se pueden usar otras fórmulas en el texto para variar el comportamiento del paquete de acrónimos:
% Por ejemplo:
% acf{} hace que siempre aparezca el texto completo del acrónimo correspondiente. (fórmula larga y corta)
% acl{} hace que Solo aparezca la fórmula larga
% acs{} hace que aparezca obligatoriamente la fórmula corta
% acp{} incluye el plural del acrónimo.
% acs{} hace que aparezca la versión córta del acrónimo.
% acresetall{} resetea todos los acrónimos de forma que se establecen como “no usados”.
% acused{} marca el acrónimo como “usado”.

A lo largo del documento serán utilizadas una serie de abreviaturas con el fin de
hacer más cómoda su lectura. Todos los términos están indicados a continuación: 

\begin{itemize}

\acrodef{TFG}{Trabajo Final de Grado}
\acrodefplural{TFG}[TFG]{Trabajos Finales de Grado}
\item \textbf{TFG:} Trabajo Final de Grado

\acrodef{PC}{Personal Computer}
\acrodefplural{PC}[PC]{Personal Computers}
\item \textbf{PC:} Personal Computer

\acrodef{RTS}{Real Time Strategy}
\item \textbf{RTS:} Real Time Strategy

\acrodef{RPG}{Rol Play Game}
\acrodefplural{RPG}[RPG]{Rol Play Games}
\item \textbf{RPG:} Rol Play Game

\acrodef{IA}{Inteligencia Artificial}
\acrodefplural{IA}[IA]{Inteligencias Artificiales}
\item \textbf{IA:} Inteligencia Artificial

\acrodef{NPC}{Non-Player Character}
\acrodefplural{NPC}[NPC]{Non-Player Characters}
\item \textbf{NPC:} Non-Player Character

\acrodef{AoE}{Age of Empires}
\acrodefplural{AoE}[AoE]{Age of Empires}
\item \textbf{AoE:} Age of Empires

\acrodef{TaB}{They are Billions}
\acrodefplural{TaB}[TaB]{They are Billions}
\item \textbf{TaB:} They are Billions

\acrodef{ECS}{Entity-Component-System}
\acrodefplural{ECS}[ECS]{Entity-Component-System}
\item \textbf{ECS:} Entity-Component-System

\acrodef{GDD}{Game Design Document}
\acrodefplural{GDD}[GDD]{Game Design Documents}
\item \textbf{GDD:} Game Design Document

\acrodef{MVP}{Mínimo producto viable}
\acrodefplural{MVP}[MVP]{Mínimos productos viables}
\item \textbf{MVP:} Mínimo producto viable

\end{itemize}